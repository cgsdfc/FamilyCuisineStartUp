\chapter{产品项目}

\section{产品前景}

\subsection{背景介绍}
亲子关系是维护社会和谐稳定的重要一环,是每一个家庭不得不面对的话题;良好
的亲子关系是孩子健康成长的基石,也是父母安心工作的基础。维护良好的亲子关系需
要孩子和父母在日常生活的方方面面的共同努力,而用餐时段则是非常难得的机会。

餐饮企业在如今的中国遍地开花。我国基本上所有的餐馆都围绕菜品特色做文章,
比如八大菜系的鲁菜、川菜、粤菜、苏菜、闽菜、浙菜、湘菜、徽菜和新式菜,比如火
锅店、洋快餐、拉面店等等。这些餐馆基本上除了提供菜品和就餐环境,并没有为就餐
人群作特殊的考虑。一些洋快餐店确实提供了儿童活动区,但是并没有为家长做考虑。

为了弥补已有餐饮企业的不足,我们为家长和孩子创造一个温馨舒适的就餐环境,
提供老少咸宜的菜品以及家庭活动的道具,让孩子和家长在就餐的同时多多沟通,增加
感情。

\subsection{亲子主题餐饮的概念}
亲子主题餐饮的概念是在一顿饭的时间里,为家长孩子提供既健康美味又有趣的就
餐体验,进行愉快的亲子交流和互动。我们的目标人群是 3-12 岁的儿童和重视下一代
培养和教育的家长。我们提供亲子 DIY 烹饪,儿童餐厅,生日派对等等活动。我们的服
务人员都具有幼教资格。更重要的是,我们有专业的健康膳食研究团队,研制出适合不
同年龄段儿童生长发育的、适合产后妈妈身体恢复的营养美食。

\subsection{亲子主题餐饮的发展空间}
亲子主题餐饮和传统的餐饮行业存在竞争关系,必须提供差异化的产品和服务和合
理的价格才能有竞争力。为此,我们必须抓住儿童和父母心理上的特点,满足各自的需
求。父母希望孩子吃健康的食品,收获一些对成长有利的知识;孩子希望见到新奇的食
物,喜欢动手尝试,获得成就感和归属感。亲子主题餐饮可以有针对性的提供各种食品
DIY,让孩子在动手动脑中制作出精美的食物;父母则从旁帮助,从而和孩子交流和合
作,促进感情的交流。为了丰富业务,还可以提供制作示范,让年轻妈妈学做健康的幼
儿膳食。如果针对亲子特有的需求提供差异化的服务,那么亲子主题餐厅将获得传统餐
饮行业所没有的发展空间。

\section{服务概述}

\subsection{服务介绍}
服务特色:童话主题餐,亲子 DIY,亲子娱乐游戏,生日派对。

\subsection{服务优势}
本项目的特色主要有:特色童餐、主题装潢、亲子活动和特色自助餐。餐厅的特色
菜品为儿童餐,兼顾了美味、健康和新颖。提供适合儿童好奇心理的和照顾了儿童消化
系统的菜品,以精美、色彩斑斓和营养均衡为特色。在装修上走类似迪士尼乐园的故事
主题装修路线,有适合不同年龄段儿童,从童话故事,小说,电影中提取的经典场景,
如星球大战主题;冰河主题;小黄人主题等。提供丰富多彩的亲子活动,包括户内的蛋
糕 DIY、亲子陶艺、英语沙龙等项目和户外的亲子拔河、两人三足、亲子吹乒乓球等项
目。利用用餐前后或者节假日等时段开展。其中户外活动和餐饮联动,参加户外活动的
顾客可以获得就餐优惠。此外还承接各种节日派对。除此之外,餐厅还提供品种丰富、
精选各国特色美食的自助餐,为不同顾客提供了不同的选择。

\subsection{服务前景}
由于目前亲子互动主题餐饮行业还属于新兴产业,现有企业不多,且实施难度系数
比小,因而门槛低,要进入市场相对容易;但是市场需求会随着社会经济的发展与日俱
增,在未来将会有较多的竞争者。面临的主要风险是:铺租、水电和原材料的成本上升
压缩利润空间。新意不足或者创新赶不上顾客厌倦的速度。成功的商业模式或者营销策
略被同行模仿导致流量减少。由于经济增长速率放缓,愿意外出就餐的人急剧减少。

\subsection{研究与开发}
\begin{description}
        \item[短期目标:] 巩固现有服务的质量。
        \item[中长期目标:] 丰富产品种类,提供不同层次的服务。
\end{description}

\subsection{未来产品与服务规划}
我们将亲子互动餐饮的年龄范围提高,让其适应青少年的孩子和他们的家长。针对
青少年存在的叛逆心理,提供对症下药的服务,促进家长和孩子缓和矛盾,增进相互理
解。

\section{生产技术管理}

\subsection{生产场地管理}
%% Figure company location
基于以下的考虑,公司的第一家分店选址在北京市海淀区五道口购物中心:
\begin{enumerate}
                \item 市中心较近的人口密集区,交通比较便捷。
                \item 靠近教育机构,方便家长带孩子来用餐。
                \item 人流密集,有助于公司的宣传。
\end{enumerate}

\subsection{原材料的采购与管理}
公司的原材料主要是未加工(半加工)的食品原材料、互动所需要的教具、服务人
员的服装等等。本着健康至上的原则,公司对食品材料进行严格的质量控制,绝对不售
卖过期或者变质食品,为消费者的健康负责。公司还将定期消毒教具,更换老旧的设施,
保持室内装潢的整洁,时尚和舒适。服务人员穿着整齐划一的服饰。厨师等厨务人员进
入工作岗位前要进行身体的消毒;厨房定期进行消毒。确保给家长和孩子一个卫生良好
的用餐环境。

\subsection{产品质量管理}
\begin{description}
\item[一、 服务流程:]
作为一家亲子主题餐馆,日常的经营以提供亲子互动的用餐体验为主。我们的组织
结构分为前台、后台、管理层三个部门;前台有分为服务部和亲子互动部;后台分为厨
房和场景布置部。其中,管理层负责管理监督和协调前后台各项事务,还负责菜品的研
发和解释;前台负责招待来客;后台负责生产菜品。各部门的员工可以向管理层直接反
映意见,确保渠道畅通。

\item[二、各部门人员分配:]
人员的分配参见表~\vref{table:人员分配}。
\begin{table}
\centering
\caption{技术研发人员分配}
\label{table:人员分配}
\begin{tabular}{|c|c|c|}
\hline
类别&人数&主要工作\\ \hline
大厨&2&烹饪高级菜肴\\ \hline
中厨&5&烹饪中级菜肴\\ \hline
下厨&10&烹饪一般菜肴\\ \hline
洗碗工&5&洗碗\\ \hline
厨杂工&5&食材预处理,清理厨房\\ \hline
食品工程师&3&研发菜品\\ \hline
前台服务人员&10&点菜,传菜,结账\\ \hline
互动组织人员&10&组织,开展亲子互动\\ \hline
前台经理&3&管理前台事务\\ \hline
后台经理&2&管理后台事务\\ \hline
场景布置人员&3&布置派对场景\\ \hline
\end{tabular}
\end{table}
\end{description}

\subsection{成本控制}
成本控制主要在这几个方面展开:
\begin{enumerate}
\item 食品加工。和高档食品供应商形成合作伙伴关系,以大批次的订货获取较为优惠
的价格。
\item 服务人员薪酬。由于亲子主题餐饮对服务人员的素质要求较高,将会以高薪聘请
高素质人才并加以培训;对于持幼教资格证的员工还有特殊待遇。
\item 菜品研发经费。为了提出创新、健康而又美味的菜肴,我们必须聘请营养学、幼
儿学等专业人士,在有相关经验的项目经理的带领下进行研发。我们需要支付研发人员
的薪酬较高,加上研发经费,实验成本等,将会在整体支出中占据较大的份额。
\item 各项杂费,铺租水电等。
我们将秉持质量至上,服务至上的原则,在不牺牲服务质量的前提下,尽可能的节
约整体支出。我们将落实一下措施:和食品供应商达成合作伙伴关系,建立阶梯式的薪
酬体系,确保资金透明,进行财务公开。
\end{enumerate}
我们将秉持质量至上,服务至上的原则,在不牺牲服务质量的前提下,尽可能的节
约整体支出。我们将落实一下措施:和食品供应商达成合作伙伴关系,建立阶梯式的薪
酬体系,确保资金透明,进行财务公开。

\subsection{管理模式}
高效的生产效率离不开科学的管理制度。每个企业都会根据自身发展的特点制定出
适合自己的管理制度。公司的基本管理政策是:
\begin{enumerate}
\item 下级服从上级。
\item 跨部门管理要征得管理层同意。
\item 各项决策应参考顾客的反馈意见。
\item 重大人员变动,如辞退、升职、降职、录用,须由和人员变动有关的部分的负责
        人和管理层人事部门代表组成的临时委员会进行商讨后决定。
\item 公开财务流水明细和薪酬各项明细。对公示内容有疑义者,可以向管理层提出复
        核。
\end{enumerate}
