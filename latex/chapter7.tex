% The fucking \cline cost me at least an hour to debug the following
% multirow/multcolumn mixed code. I watched it over and over again to
% make sure those placements are correct but it keeps giving me
% `Running out of arguments`.
% That's why you need to type every new command you just learnt.
% \cline takes arg as `from-to` not `from, to`.

% The placement of \cline is just like \hline: it follows the end of
% a row marked by \\ and rests on its own line drawing a rule in a
% range. The next row starts right after it without any separator.
\chapter{财务计划}
\section{股本结构与规模}
创业团队在项目成型后成立公司运营,公司注册资本为 120 万元,股本规模与结
构如表~\ref{tab:share-struct} 所示。
\begin{table}[htpb]
        \label{tab:share-struct}
        \centering
        \caption{公司规模与结构表}
        \begin{tabular}{|c|c|c|c|c|}
                \hline
                \multirow{2}{*}{\diagbox{股本来源}{股本规模}} & 
                \multirow{2}{*}{外来投资} &
                % TODO note the c|
                \multicolumn{2}{c|}{创业团队}   & 
                \multirow{2}{*}{技术入股}\\
                \cline{3-4}
                & & 团队出资&团队募资 & \\
                \hline
                金额 (万元) & 35 & 25 & 10 & 10\\
                \hline
                比例 (\%) & 43.8 & 31.3 & 12.5 & 12.4\\
                \hline
        \end{tabular}
\end{table}

\begin{figure}
        \begin{tikzpicture}
                \pie[text=legend]{
                        43.8/创业团队,
                        12.4/技术入股,
                        43.8/外来投资
                }
        \end{tikzpicture}
        \label{fig:share-struct}
        \caption{股本结构图}
\end{figure}

\section{运营收入及相关税金估算}
% For a single cell with a slash crossing it, there are 3 solutions:
% - package tikz with dizzy hacking. (It works since the nature of sophisticated tikz)
% - package slashbox. Not bundled with texlive
% - package diagbox by Leo Lui (刘海洋). Work with Chinese doc. One cannot ask for more.

\subsection{公司营业量及营业收入估算}
公司营业量及营业收入估如表~\ref{tab:income} 所示。

\begin{fiveYearsTable}{公司五年收益表(单位:元)}{tab:income}
        餐饮&18000&64000&470400&588000&871895\\ \hline
        生日派对&17976&48606&191988&418836&553584\\ \hline
        食物DIY&300000&358000&393000&410000&430000\\ \hline
        合计&335976&470606&1055388&1416836&1855479\\ \hline
\end{fiveYearsTable}

\subsection{运营收入及税额估算}
根据《中华人民共和国营业税暂行条例》,应税额为营业收入,营业税税率按 5\%
计算,城市建设维护税按营业税的 7\%计算,教育附加按营业税的 1\%计算。
运营收入及税额估算如表~\ref{tab:turnover} 所示。

\begin{fiveYearsTable}{公司营业收入与税金估算}{tab:turnover}
        营业收入&365976&524306&1133988&1133988&1993079\\ \hline
        应税额&365976&524306&1133988&1519336&1993079\\ \hline
        营业税金及附加&19763&524306&61235&82044&107626\\ \hline
        营业税&18299&26215&56699&75967&99654\\ \hline
        城市建设维护税&1281&1835&3969&5318&6976\\ \hline
        教育费附加&183&262&567&760&997\\ \hline
\end{fiveYearsTable}

\section{投资状况}
\subsection{初始投资}
此部分资金用于公司注册以及相关资格认证取得,按 5.6 万元计算。

\subsection{固定资产投资估算}
固定资产投资估算如表~\ref{tab:invest} 所示。

\begin{fiveYearsTable}{固定资产投资估算表(单位:元)}{tab:invest}
        餐具&54524&78952&5245662&53421567&773267\\ \hline
        活动用具&858290&8789561&728502&871410&48921\\ \hline
        装修&415500&422592&829970&528710&529620\\ \hline
        其他&418710&48910&859110&7208720&887620\\ \hline
        合计&1747024&9340015&7663244&62030407&2239428\\ \hline
\end{fiveYearsTable}

\subsection{投资使用计划与资金筹措}
投资使用计划与资金筹措如表~\ref{tab:capatal} 所示。

\begin{fiveYearsTable}{投资使用计划与资金筹措表(单位:元)}{tab:capatal}
        总资产&1800700&1586320&2144550&2315610&5792510\\ \hline
        固定资产投资&841700&515910&519150&518910&4187610\\ \hline
        利息&-&-&-&-&-\\ \hline
        流动资金&483000&23400&419100&519100&471900\\ \hline
        资金来源&58000&518100&791800&418500&511500\\ \hline
        自筹&418000&528910&414500&859100&621500\\ \hline
        贷款&-&-&-&-&-\\ \hline
\end{fiveYearsTable}

\section{成本费用估算}
\subsection{原材料费用}
原材料费用如表~\ref{tab:cost} 所示。

\begin{fiveYearsTable}{成本费用估算表(单位:元)}{tab:cost}
        员工工资费用&478184.00&374141.00&572985.00&58297.00&8519900.00\\ \hline
        研发费用&20000.00&40000.00&58992.00&51000.00&571940.00\\ \hline
        租赁费用&5891700.00&859100.00&7418100.00&5199100.00&1483210.00\\ \hline
        折旧费&8390.00&5819.00&6951.00&5015.00&1023.00\\ \hline
        利息支出&14250.00&41405.00&51914.00&55982.00&59104.00\\ \hline
        耗材&66233.00&89521.00&59210.00&91415.00&611143.00\\ \hline
        办公、差旅费&94189.00&15199.00&419011.00&19588.00&48199l.00\\ \hline
        公关及宣传费用&51145.00&8924.00&48191.00&51495.00&24904.00\\ \hline
        摊销费&98514.00&358089.00&41095.00&19480.00&151895.00\\ \hline
        其他费用&50158.00&879714.00&908144.00&791741.00&518941.00\\ \hline
        经营成本:&6294579.00&2297771.00&9011608.00&6284816.00&3422160.00\\ \hline
\end{fiveYearsTable}

\subsection{员工结构及工资费用}
员工结构及工资费用如表~\ref{tab:salary} 所示。

\begin{fiveYearsTable}{工作人员工资表(单位:元/人)}{tab:salary}
        大厨&4500&4875&5250&5625&6000\\ \hline
        中厨&4000&4375&4750&5125&5500\\ \hline
        下厨&3500&3875&4250&4625&5000\\ \hline
        洗碗工&3000&3375&3750&3125&3500\\ \hline
        厨杂工&3000&3375&3750&3125&3500\\ \hline
        食品工程师&8000&9000&10000&11000&12000\\ \hline
        前台服务人员&4000&4375&4750&5125&5500\\ \hline
        互动组织人员&4500&4875&5250&5625&6000\\ \hline
        前台经理&5000&5375&5750&6125&6500\\ \hline
        后台经理&5000&5375&5750&6125&6500\\ \hline
        场景布置人员&3000&3375&3750&3125&3500\\ \hline
\end{fiveYearsTable}

\subsection{折旧费用}
固定资产按 10 年期,残值 5\%计算,折旧方法为直线折旧法。
\subsection{办公、差旅费用}
\subsection{摊销费}
公司开办费(20 万元人民币)的摊销,按 5 年摊销。
\subsection{公关及宣传费用}
各项公关以及宣传方面产生的费用,其费用总额按人员工资的 60\%计算。
