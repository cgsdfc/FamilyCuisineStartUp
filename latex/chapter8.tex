\chapter{财务分析}
\section{公司利润表及现金流量表}
\subsection{公司利润估算分析}
公司利润表详见策划附录 2;公司的利润总额到第五年达到 96 余元。公司所得税税
率按 25\%计算。税后利润到第五年达到 72 万余元。盈余公积金(含公益金)按可分配利
润的 15\%提取。从投资第 4 年起按可供投资者分配利润的 40\%分配利润。
\subsection{公司现金流量分析}
公司五年期利润表详见策划附录 3;从现金流量表可以直接看出公司现金流量从第
二年开始转正现金流量平稳增长,第五年累积税后现金流量达到 141 万余元。

\section{公司盈利能力分析}
\subsection{静态盈利能力分析}
(1) 项目的总投资收益率税后利润与项目的总投资的比值,即 90\%。
(2) 项目的资本金净利润率本金净利润率采用达产期的净利润与项目的自有资金
的比值,即 144\% ,回报率相当高。
(3) 投资回收期
税前:2.60 年。
税后:2.70 年。
\subsection{动态盈利能力分析}
项目的折现率为 10\% (综合考虑无风险收益率和项目本身的风险程度确定),以下指
标中都是按照 5 年计算。

%% Formula
项目的折现率为 10\% (综合考虑无风险收益率和项目本身的风险程度确定),以下指
标中都是按照 5 年计算。

%% Formula
(2)项目投资财务内部收益率(
)项目投资财务内部
收益率(所得税前)为 317\%。项目投资财务内部收益率(所得税后)为 277\%。因此从动态
指标角度分析,项目投资财务净现值无论从税前还是税后角度来说都大于零,且数额较
大,说明项目的盈利能力很强。同时,项目投资财务内部收益率比财务基准收益率的 10\%
要高很多。
综上所述,项目具有很强的盈利能力。
\section{敏感性分析}
敏感性分析选用了营业收入和经营成本作为影响因素,来分析这些因素的变化对主
要经济指标的影响程度, 图表如下:
% TODO 表格 12 敏感性分析图表
% TODO 图 8.1 成本与收益变动敏感性比较分析图
在此图中可以清楚地看出 NPV 对于营业收入的变化更敏感。在经营过程中要对此
多加注意,提升营业收益,促进企业向更好的方向发展。

