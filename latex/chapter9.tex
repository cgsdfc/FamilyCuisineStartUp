\chapter{投资分析}
\section{投资结构及分析}
\subsection{注册资本结构与规模}
公司计划注册资本 120 万。
\subsection{投资回报}
根据对未来五年公司经营状况的预测,公司能保持较高的利润增长。拟从第三年起
每年从净利润中提取 20\%作为股东的分配红利。公司净资产收益率略高于总资产报酬率,
这是一个很好的征兆,说明公司股东价值不断增长,有利于筹资渠道的不断完善。公司成
立初期,规模较小,报酬率基本维持在 60\%,随着公司的不断发展壮大,2 年实行自行生
产后资产回报大幅上扬,回报率会较高。
%% TODO 图 9.1 产品盈利能力预计分析

\section{风险分析及防范措施}
主要假设: 物流公司、室内设计公司和食品供应商的信誉足够好, 装修、员工招募、
试运营在 2 个月内完成;公司的首家分店选址在北京市海淀区五道口购物中心。
\subsection{风险管理系统}
公司的生存、发展始终暴露在各种内、外部风险中,风险无时不在,无处不在。如果
对内、外部风险认识不清,控制不当,必然会影响企业的经营发展。公司建立风险管理系
统,对风险进行识别、分析和控制。
风险系统流程如下图:
% TODO 图 9.2 风险系统流程图
% TODO 图 9.3 风险分析图
\subsection{市场风险}
在公司的运作中,因市场突变、人为分割、竞争加剧、通货膨胀或紧缩、原材料供
应等事先未预测到的风险,可能导致市场份额的急剧下降。
\subsection{运作风险}
公司内部混乱、资产负债率高、资金流转困难、三角债困扰、资金回笼慢、资产沉
淀,造成资不抵债的困境。
\subsection{人力资源风险}
公司对其管理人员、销售人员任用不当没有充分授权,或精英流失,造成公司人力资
源的损失。
\subsection{公关风险}
如劳资纠纷、法律纠纷、使企业公信力和美誉度急剧下降。

\section{风险投资资金的撤出方式}
风险投资一般通过资本与管理投入,在企业的成长中促进资本增值,并且在退出时
实现收益变现。但风险投资的成功率非常低,更是面临着所投资的企业破产清算的危险。
根据一项关于美国 13 个风险投资基金的分析研究表明,风险投资总收益的 50\%来自于
6.8\%的投资,总收益的 75\%来自于 15.7\%的投资。真正能为风险投资者带来收益的投资
项目还不到 1/4。所以,如何减少风险投资项目的损失,确保成功投资的收益顺利回收对
于风险投资者来说至关重要。而缺乏便捷市场化的退出渠道更是制约我国风险投资事业
发展的最大障碍。目前,国内风险投资的退出方式一般有以下几种:首次公开上市(IPO)、
收购和清算、剩余利润分红、股权转让、二次出售等方式。鉴于公司形式和具体状况,
本公司选择了以下几种风险资本撤出方式:
1)国内中小企业版上市:本公司会在公司运营 5-7 年上市,因为此时公司已发展到
相当大的规模,公司已经连续盈利了三年,公司的净资产,股本结构,营业收入,经过改组
后的公司制度都符合了国内二板市场上市的条件。从投资的时间和公司发展的角度考虑,
公司经过了导入期和成长期,已完成一部分新产品和相关产品的开发,发展趋势很好,此
时上市对公司和风险投资家来说,都是有利的。
2)海外二板市场上市:本公司属于高科技产业,同时它也属于有发展前景和增长潜
力的中小型高新企业,可争取在香港二板市场上市,与国内相比,在海外上市条件更宽松,
较适合有较好发展前景的科技型企业上市,这为本公司提供了条件。
3)兼并与收购:考虑到上市存在着股票价格下跌的风险,且公开上市后,需要一段时
间,风险投资才能完全推出。那些不愿承担风险和受种种约束的风险投资家可以选择此
种方式,对于风险投资家来说,此种方式是有吸引力的,因为这种方式可以让他们立即收
回投资,也使得其可以立即从风险企业中退出。兼并和收购包括两种:即一般收购和“第
二期收购”。一般收购主要指公司的收购与兼并;“第二期收购”指由另一家风险投资
公司收购。本公司将会实行第一种方式收购风险投资家的股份,因为本公司在第 4-5 年
已有较高的利润和现金,本公司完全有能力收回股份,从而实现真正意义上对公司的管
理。
4)股份转让:股份转让是风险资本退出的另一条途径。当本公司发展到一定程度,
要想再继续发展就需要大量的追加投资,风险投资者有可能不愿继续投资。此时,企业的
良好发展态势会吸引另外一些风险投资家或者银行资本的进入。那么,原有的风险投资
家可把手中的股份转让给这些人,从而实现自己资金的退出。上述是针对不同风险投资
家不同时期可选择资金退出的方式,当风险投资家期望获得高收益,且相信企业的发展
前景,可选择公开上市;当风险投资家希望尽快退出投资更有收益的行业,可选择股权转
让。同时当本公司有较高的利润,且能够完全控制企业时,本公司会通过收购的方式收回
那些不愿继续投资的投资家的股份,从而实现他们的退出。
