\chapter{机遇与风险}
%% TODO These are all itemize with big black dot as marker
%% like:
% ● 二孩政策导致我国新生人口呈现增长趋势,形成了潜在的巨大市场。
% ● 亲子餐厅迎合父母和孩子交流情感,符合孩子好奇的天性,能吸引不少家长偕
% 同孩子用餐。
% ● 目前亲子餐厅属于新兴市场,竞争者较少。
\section{机遇}
● 二孩政策导致我国新生人口呈现增长趋势,形成了潜在的巨大市场。
● 亲子餐厅迎合父母和孩子交流情感,符合孩子好奇的天性,能吸引不少家长偕
同孩子用餐。
● 目前亲子餐厅属于新兴市场,竞争者较少。


\section{外部风险}
● 传统餐饮行业坐拥客源、资金、和人们的消费习惯息息相关,是强有力的竞争
对手。
● 未来,随着亲子餐厅的兴起,行业可能涌入大量竞争者。
● 针对 3-13 岁的儿童,生命周期短。
● 难以克服顾客喜新厌旧的心理;亲子餐厅的创新之举容易落入俗套。

\section{内部风险}
● 公司要同时做好传统餐饮行业必须重视的食物安全、生产安全、资金周转等工
作和亲子餐厅独有的活动组织、特殊装潢、菜品营养考核和服务人员资格审核等方面
——容易顾此失彼。
● 经验时要格外注意儿童的生命财产安全。
● 需要长期保持菜品创新和活动创新以吸引顾客。

\section{解决方案}
● 菜品走精致、健康和营养路线,装修走童趣、新奇、轻松路线,活动走安全、
有趣、益智路线。形成和传统餐饮行业对比鲜明的特色。
● 依靠优质的食品、贴心的服务和合理的价格形成强大的用户群;和当地政府在
儿童公益事业上形成合作伙伴关系;制造品牌效应;捍卫并且扩大市场份额。
● 公司可以同时经验其他母婴类商品,在客流低谷期提供营业额的保障。
● 在产品研发团队加大资金投入,保持创新的力度,不断推出新菜品,新活动。
● 加强内部建设,完善规章制度,确保各项事务有条不紊的进行。
● 加强员工的安全意识,特别是对没有自我保护能力的低龄儿童,强化他们的安
保技能;培养员工应对突发事件的能力。



