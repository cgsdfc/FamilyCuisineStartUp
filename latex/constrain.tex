就治理结构而言,仅有激励措施是不可行的,有时激励措施是好的,但也存在人力
资本运作不好的问题。 因此,在建立激励机制的同时,还必须建立约束机制。 人力资
本的约束机制大致可以分为内部约束和外部约束两个方面。

\subparagraph{内部约束}
内部约束是公司和人力资本之间的约束,以及各方之间的约束。
从国际上看,这种内部制约主要有四个方面的约束措施。该公司的章程是有约束力的。
当人力资本进入一家公司并且公司受到制约时,第一个约束就是公司的宪法规定,这意
味着公司的每个人都必须服务并提交给公司章程。合同限制。企业中的任何人力资本都
必须签署非常详细的合同。这种合同必须表现出对公司商业秘密的保护,对技术专利的
保护以及对竞争力的保护。偏好约束。所谓的偏好约束,就是我想约束你,首先要考虑
你的偏好是什么。如果你想实现自己的经营理念,而不是更多的钱,那就用它来强化你,
来约束你。制度约束。所谓制度约束是指高度重视改善企业的最高决策机构。人力资本
与企业之间的摩擦和矛盾必须发生,成人与机构之间的矛盾应该得到发展。企业与人力
资本之间的矛盾之所以不能转化为人与人之间的矛盾,是因为人与人之间的摩擦难以使
人力资本受到正常的约束。更重要的是,这种限制往往会增加。在个人的好恶之中,往
往需要将人与成人之间的摩擦转化为组织间的摩擦,这必须高度重视公司决策机构的完
善。

\subparagraph{外部制约因素}
所谓的外部制约因素实际上是社会制约因素,即人力资本形成
中的社会制约因素。该约束通常具有以下几个方面:法律约束。公司全体员工必须遵守
有关法律法规,诚实守信,依法行事。道德约束。每个阶层都应该有自己的职业道德,
所以人力资本也应该受到道德上的制约。市场约束。人力资本作为一种资本流经人力资
本市场。这种市场在遏制人力资本方面应该起到非常重要的作用。社会群体约束。所谓
的社会群体约束意味着作为人力资本,它应该有自己的非政府组织,因为民间社会组织
实际上是市场约束和道德约束之间非常重要的约束。媒体限制。媒体限制必须遵守促进
某些新闻的标准。应该有利于企业的发展。它必须考虑企业的负担能力和收益,不应该
炒作。
