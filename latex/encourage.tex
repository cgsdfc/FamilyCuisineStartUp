\paragraph{保持良好的薪酬体系}
餐饮公司的员工教育水平相对较低,一般更关心物质激励。 员工用自己的报酬来衡量
自己的价值,相对工资水平也影响员工的公平感。 尽管薪酬和福利不足以激励员工,
但他们是必要的工具。
\subparagraph{ 体现公平 }
当员工对薪酬制度感到不公平时,可能会采取一些负面的反应,如减少责任和辞职。
公平的补偿制度不仅包括内部公平,还包括外部公平。
\subparagraph{ 激励 }
餐饮公司是相对低收入的公司。因此,他们可以采用高度稳定的薪酬模式,提高基
本薪酬和保险福利的比例,降低绩效薪酬比例,使员工感到安全。绩效工资必须保持可
用,评估指标必须具备客户满意度,劳动时间和其他关键指标。公司在指定激励制度时
必须有明确和一致的原则,并且应该有统一和可执行的准则作为基础。激励头寸之间的
工资差距必须有一定的基础和全面合理。为降低员工离职率和留住核心人才,公司可以
采取具有长期激励效应的薪酬体系,如员工持股和股票期权(ESO)。

\paragraph{补充使用心理奖励}
奖励是员工需求的满足。员工的需求多种多样,因此激励措施各不相同。应允许物
质激励和精神激励相辅相成。
\subparagraph{ 工作目标奖励 }
目标是衡量员工的工作表现。在食品和饮料行业,管理人员应该
为员工制定服务标准或提出要求作为工作目标。
\subparagraph{ 工作流程奖励 }
管理者应该经常向员工灌输正确的专业观点,并经常赞扬和鼓励
员工。另外,员工也应该被允许去做他们最喜欢的工作。他们可以通过轮换或张贴来丰
富自己的角色,这样员工就可以找到自己的兴趣爱好,并尽力让他们在自己最喜欢的职
位上做自己喜欢的工作。

\paragraph{工作完成激励}
对于一些优秀的员工来说,例如某种创新菜的销售量很高,或者
服务得到客户的赞扬,餐厅经理或厨师可以给员工一些小礼物,例如配方,新配料。工
作服,一个美丽的新厨房,等等。但是,它必须是实用的而不是浮华。
\subparagraph{ 荣誉奖励 }
对于员工来说,一方面他们获得较高的评价和尊重,就会产生一种成就感和心理满
足感和自我实现感。另一方面,荣誉也意味着未来一个人获得更好的收入的可能性,因
为一个人过去的工作的良好声誉可能会使他在当前的工作企业中获得更高的印象。
\subparagraph{ 建立科学的人才选拔机制 }
适当运用内部促进,坚持公开,公平,公正的原则,
为内外部人才吸引和选拔实际人才提供平等机会,逐步摆脱“家长制”管理模式。
\subparagraph{ 完善培训机制 }
员工培训应侧重于公司服务的特点。它不仅要强调企业服务知识的培训,还要强调
人际交往技能的培养,包括沟通技巧,冲突解决能力和跨文化敏感意识。考虑到餐饮业
的长远发展,餐饮业的安全性和员工忠诚度的培养,公司内部培训是一种符合成本核算
原则的明智举措。
